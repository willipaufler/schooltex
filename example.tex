\documentclass[12pt, solution=false]{sheet}
%default geometry
%\usepackage{geometry}\geometry{a4paper, left=20mm,right=20mm,bottom=25mm,top=10mm} % defines margins and geometry

\usepackage{school}

\date{Dezember 2019}
\renewcommand\lecture{Mathe, Klasse 8}
\title{Leistungskontrolle}
\renewcommand\name{Herr Paufler}

\begin{document}


\begin{task}{Schriftliche Addition}
Berechne folgende Additionsaufgaben.\\

\threesubtasks{\add{3322}{123}}{\add{32}{723}}{\add{2234}{1982}}

\threesubtasks{\add{6427}{2133}}{\add{22}{19}}{\add{985}{762}}
\end{task}
%
\begin{solution}

\threesubtasks{3445}{755}{4216}

\threesubtasks{2560}{51}{1757}
\end{solution}

%
%-------------------------Aufgabe----------------------------%
%
\begin{task}{Geometrische Figuren}
Berechne den Flächeninhalt der folgenden Figuren.

\threesubtasks{
\begin{tikztask}
   \pic {my circle={20mm}{71cm}};
\end{tikztask}
}{
\begin{tikztask}
  \draw (0,0) -- ++(40mm,0) node [midway, below] {37cm} -- ++(0,20mm) node [midway, right, anchor=north, sloped] {21cm} -| cycle;
\end{tikztask}
}{
\begin{tikztask}
  \draw (0,0) coordinate (a) -- ++(20mm,0) node [midway, below] {10cm} -- ++(-10mm,25mm) coordinate (c) -- cycle;
  \draw [dashed] (a -| c) -- (c) node [pos=.35, right, anchor=south, sloped] {8cm};
  \draw (a -| c) ++ (0,5pt) -| ++(5pt,-5pt);
\end{tikztask}
}
\end{task}

%
%-------------------------Aufgabe----------------------------%
%
\begin{task}{Kopfrechnen}
Ein paar einfache

\threesubtasks{
2+2\\
3+3
}{
17+2\\
3+29
}{
$3\cdot 16$\\
3+3
}

\end{task}

%
%-------------------------Aufgabe----------------------------%
%
\begin{task}{Funktionswerte}
Vervollständige die Tabelle für die Funktion $f(x)=x^2+1$
\begin{table}[ht]
        \centering
        \begin{tabular}{ C{15mm} || C{15mm} | C{15mm} | C{15mm} | C{15mm} | C{15mm} | C{15mm}}
           $x$ & 1 & 2 & 5 &  & 11 & 13 \\
            \hline
            $y$ &  &  &  & 82 & & 
        \end{tabular}
    \end{table}
\end{task}


%
%-------------------------Aufgabe----------------------------%
%
\begin{task}{Multiplikation und Addition}
Ergänze die fehlenden ganzen Zahlen. Gestrichene Linien bedeuten Addition, durchgezogene Multiplikation.\\
\threesubtasks{
\begin{tikztask}
	\node[circ](a) at (0,0) {18};
	\node[circ](b) at +(-60: 1.5) {2};
	\node[circ](c) at +(-120: 1.5) {};
	\node[circ](d) at +(-90: 2.6) {4};
	\node[circ](e) at +(-120: 3) {};
	\draw [dashed,thick,->]  (b) -- (a);
	\draw [dashed,thick,->]  (c) -- (a);
	\draw [thick,->]  (e) -- (c);
	\draw [thick,->]  (d) -- (c);
\end{tikztask}
}{
\begin{tikztask}	
	\node[circ](a) at (0,0) {18};
	\node[circ](b) at +(-60: 1.5) {2};
	\node[circ](c) at +(-120: 1.5) { };
	\node[circ](d) at +(-90: 2.6) {4};
	\node[circ](e) at +(-120: 3) { };
	\draw [thick,->]  (b) -- (a);
	\draw [thick,->]  (c) -- (a);
	\draw [dashed,thick,->]  (e) -- (c);
	\draw [dashed,thick,->]  (d) -- (c);
\end{tikztask}
}{
\begin{tikztask}
	\node[circ](a) at (0,0) {};
	\node[circ](b) at +(0: 2) {};
	\node[circ](c) at +(-60: 2) {};
	\path[->]
		(a) edge [bend right=-45] node {} (b)
		(c) edge [bend right=-45] node {} (b);
	\path[dashed,->]
		(a) edge [bend right=45] node {} (c)
		(b) edge [bend right=-45] node {} (c);
\end{tikztask}
}
\end{task}
%
\begin{solution}
\threesubtasks{
\begin{tikztask}
	\node[circ](a) at (0,0) {18};
	\node[circ](b) at +(-60: 1.5) {2};
	\node[circ](c) at +(-120: 1.5) {9};
	\node[circ](d) at +(-90: 2.6) {4};
	\node[circ](e) at +(-120: 3) {5};
	\draw [dashed,thick,->]  (b) -- (a);
	\draw [dashed,thick,->]  (c) -- (a);
	\draw [thick,->]  (e) -- (c);
	\draw [thick,->]  (d) -- (c);
\end{tikztask}
}{
\begin{tikztask}	
	\node[circ](a) at (0,0) {18};
	\node[circ](b) at +(-60: 1.5) {2};
	\node[circ](c) at +(-120: 1.5) { 16};
	\node[circ](d) at +(-90: 2.6) {4};
	\node[circ](e) at +(-120: 3) {4};
	\draw [thick,->]  (b) -- (a);
	\draw [thick,->]  (c) -- (a);
	\draw [dashed,thick,->]  (e) -- (c);
	\draw [dashed,thick,->]  (d) -- (c);
\end{tikztask}
}{
\begin{tikztask}
	\node[circ](a) at (0,0) {2};
	\node[circ](b) at +(0: 2) {-4};
	\node[circ](c) at +(-60: 2) {-2};
	\path[->]
		(a) edge [bend right=-45] node {} (b)
		(c) edge [bend right=-45] node {} (b);
	\path[dashed,->]
		(a) edge [bend right=45] node {} (c)
		(b) edge [bend right=-45] node {} (c);
\end{tikztask}
}
\end{solution}

%
%-------------------------Aufgabe----------------------------%
%
\begin{task}{Zahlenpyramiden}
Wir können auch kleine Zahlenpyramiden erstellen.\\
\twosubtasks{Addiere!
\begin{pyramid}{16mm}{6mm}
	\pyOne{0}{19}
	\pyTwo{1}{}{8}
	\pyThree{2}{}{4}{}
	\pyFour{3}{}{1}{}{1}
\end{pyramid}
}{Subtrahiere!
\begin{pyramid}{16mm}{6mm}
	\pyFour{0}{105}{57}{}{12}
	\pyThree{1}{48}{25}{}
	\pyTwo{2}{}{}
	\pyOne{3}{}
\end{pyramid}
}
\end{task}


%
%-------------------------Aufgabe----------------------------%
%
\twotasks{
\begin{task}{Funktionen zuordnen}
Ordne folgenden Funktionen einen Funktionsgraphen zu und begründe deine Entscheidung!
\begin{itemize}
	\item[\asubtask] $f_1(x)=x^2-2x$
	\item[\asubtask] $f_2(x)=-x^2+2x+3$
	\item[\asubtask] $f_3(x)=-x^3-2x^2+1$
	\item[\asubtask] $f_4(x)=x^3+2x^2+2$
\end{itemize}
\end{task}
}{
\begin{plot}{-3}{5}{-6}{12}
	\addplot [color=blue,domain=-3:4, smooth,thick] {x^2-2*x};
	\addplot [color=red,domain=-3:4,smooth,thick] {-x^2 +2*x + 3};
	\addplot [color=green,domain=-3:4,smooth,thick] {x^3-2*x^2+1};
	\addplot [color=brown,domain=-3:4,smooth,thick] {x^3+2*x^2+2};
\end{plot}
}


%
%-------------------------Aufgabe----------------------------%
%
\twotasks{
\begin{task}{Integralrechnung}
Berechne die Fläche zwischen den Kurven der Funktionen
\begin{itemize}
	\item[] $f_1(x)=x^2-2x$
	\item[] $f_2(x)=-x^2+2x+6$
\end{itemize}
\begin{flushleft}
\textbf{Hinweis:} Bestimme zunächst die Schnittpunkte der Funktionen.
\end{flushleft}
\end{task}
}{
\begin{plot}{-2.5}{5}{-4}{8.5}
	\addplot [color=black,domain=-3:4, smooth,thick, name path=A] {x^2-2*x} node at (axis cs:3.4,7) {$f_1$};
	\addplot [color=black,domain=-3:4,smooth,thick, name path=B] {-x^2 +2*x + 6} node[below] {$f_2$};
	\addplot[blue!50, fill opacity=0.2] fill between[of=A and B, soft clip={domain=-1:3}];
\end{plot}
}

\end{document}
